\documentclass[12pt,a4]{article}
\usepackage[frenchb]{babel}
\usepackage[utf8]{inputenc}
\usepackage{hyperref}
\usepackage{geometry}

%%%%%%%%%% Start TeXmacs macros
\newcommand{\tmtextbf}[1]{{\bfseries{#1}}}
\newcommand{\tmtextit}[1]{{\itshape{#1}}}
%%%%%%%%%% End TeXmacs macros


\geometry{top=2cm, bottom=2cm, left=2cm , right=2cm}



\begin{document}

\title{TP intermittence}

\maketitle

Le compte rendu de ce TP devra être rendu sous forme numérique, en \textbf{pdf},
édité à l'aide d'un outil tel que \textit{Latex} ou \textit{LibreOffice}. Les figures seront
commentées et choisies selon leur pertinence.

\section{Avant de commencer}

L'intermittence est une notion très précise en physique, qui signifie que
l'on observe pour une variable aléatoire des fluctuations de grande amplitude dont la fréquence s'écarte de celle attendue pour une loi normale. Par exemple la distribution des incréments de vitesse dans un écoulement turbulent\footnote{on trouvera une abondante littérature sur le sujet à l'aide d'un moteur de recherche} ne suit pas exactement la loi normale.
Cependant dans le débat lié à
la production de puissance, et notamment dans la comparaison entre les
énergie de flux (éolien, photovoltaïque, appelé PV dans la suite) et les énergies de stock (nucléaire,
fossile) dans leur capacité à répondre à une demande de puissance
électrique, cette définition n'est pas suffisante. En effet, par exemple,
une variation parfaitement prévisible est que le PV ne produira jamais de
puissance la nuit, et ne pourra donc jamais satisfaire une demande quelle que
soit la taille du parc installé. Ainsi, il est nécessaire de tenir compte
des caractéristiques intrinsèques de la production et de la consommation et
de leur couplage, et d'identifier les notions suivantes:
\begin{itemize}
  \item Variabilité, pilotable et/ou prévisible (arret pour entretien) ou
  non (arret pour accident)
  
  \item Prévisibilité de la production (la nuit, pas de production PV)
  
  \item Fractionnabilité, c'est à dire la possibilité de n'utiliser qu'une
  partie de la puissance installée (il n'est pas possible d'arrêter une demie tranche nucléaire)  
\end{itemize}
Ces paramètres mènent à une grande confusion. Nous allons essayer dans ce
TP de percevoir les caractéristiques des différentes sources de production
de puissance électrique et d'en extraire une estimation simple de
l'intermittence.

Nous nous appuierons pour cela sur les données de production et d'échange de puissance électrique produites par RTE le Réseau de Transport Électrique français.


\section{Récupérer les données et les programmes}

\subsection{Programmes}

Rendez vous sur la page \newline
\href{https://github.com/ericherbert/TPintermittence}{https://github.com/ericherbert/TPintermittence} et téléchargez l'ensemble du projet (bouton \tmtextit{clone or download}) et extraire les
documents de l'archive. Vous y trouverez
\begin{itemize}
  \item Le dossier \tmtextit{DATA} contenant les données dont vous avez
  besoin si la phase de téléchargement des données sur le site de
  \tmtextit{RTE} n'a pas fonctionné
  
  \item L'énoncé du TP en pdf
  
  \item Des programmes \tmtextit{Python} permettant la réalisation de ce TP.
\end{itemize}
Ce TP est prévu pour être effectué sur l'application \tmtextit{JupyterHub}
de l'UFR. Vous le trouverez à l'adresse suivante
\href{https://jupy.physique.univ-paris-diderot.fr/}{https://jupy.physique.univ-paris-diderot.fr/}. Vous devez normalement pouvoir vous y connecter avec vos identifiants ENT. En
cas de problème, m'écrire.\footnote{eric.herbert@u-paris.fr}

Si vous disposez d'une installation Python 3 vous pouvez également effectuer le travail sur cette machine.

Remarque sur les données de RTE: EDF est tenu légalement d'acheter toutes la
production électrique des sources PV et eolien, donc la puissance de ces sources injectées dans le réseau est toujours la puissance maximale possible avec les installations disponibles. Inversement, les source fossiles, nucléaire et hydraulique étant pilotables, elles s'adaptent à la variation de la consommation.


\subsection{Données}

\begin{enumerate}
  \item Se rendre sur le site de RTE à l'adresse 
  \href{https://www.rte-france.com/}{https://www.rte-france.com/} puis naviguer vers l'onglet Production d'électricité par filière.
  
  \item Identifier les différentes sources de puissance, les flux exportés et importés.
  
  \item  Comparer qualitativement sur différentes periodes leur série temporelle.
  
  \item Faire un bilan comparatif des deux points précédents
  
  \item Choisir une periode spécifique et télécharger les données
  correspondantes. Noter la durée, la fréquence d'échantillonnage, les
  sources disponibles. Après avec dézippé l'archive et enregistré le
  tableau au format csv, Prendre garde également aux caractères non
  reconnus.
  
  \item Utiliser le programme \tmtextit{open\_data.py} que vous aurez modifié
  pour votre usage, pour extraire les différentes données. Noter le nom et
  la colonne de chaque mesurable (Consommation, nucléaire etc). 
\end{enumerate}


Si besoin dans le repertoire \tmtextit{DATA}, des données provenant de RTE sont
disponibles.

\section{Traitement}

Dans la suite on s'interessera uniquement aux sources de production
nucléaire, éolien et PV. L'expression \tmtextit{les 3 sources} s'y
réfère.

\subsection{Séries temporelles}

\begin{enumerate}
  \item À l'aide du programme \tmtextit{plot\_data.py}, représenter la
  dynamique temporelle de la \tmtextbf{consommation} journalière, mensuelle
  et annuelle. Ajoutez les unités et enregistrez ces figures de manière à
  pouvoir les retrouver simplement, et faites de même pour les trois sources.
  Ajoutez les unités et enregistrez ces figures de manière à pouvoir les
  retrouver simplement.
  \begin{enumerate}
    \item Sur les diagrammes annuels, repérez les différentes saisons
    
    \item Sur les diagrammes journaliers, repérez les différentes parties de
    la journée
    
    \item les éventuelles variations régionales de production de puissance
  \end{enumerate}
  \item Extraire les moyennes temporelles $\langle x \rangle$ et écart type
  $\sigma_x$ de chacune des sources sur chacune des periodes considérées.
  Quelles sont les limitations de l'emploi des quantités $\sigma_x $, $ \langle
  x \rangle$ et de la variation relative $\sigma_x / \langle
  x \rangle$, pour caractériser chacune des sources ?
\end{enumerate}

\subsection{Distribution des séries temporelles}

\begin{enumerate}
  \item À l'aide du programme \tmtextit{plot\_distribution.py}, représenter
  la variation de la distribution de la \tmtextbf{consommation} normalisée
  par sa moyenne temporelle $\frac{C (t) - C (t + \tau)}{\langle C \rangle}$,
  avec un pas de temps $\tau$ d'une heure puis de 24 heures. Faire de même
  pour chacune des 3 sources. Représenter les \tmtextit{densités de probabilité}
  (pdf) de ces fonctions en vous appuyant sur le paramètre \tmtextit{density}
  de la fonction histogramme, puis comparer ces distributions avec une loi
  normale (Gaussienne) de même moyenne et écart type. Ajouter les unités,
  noms des axes et enregistrer les figures. Vous choisirez la représentation
  (logarithmique ou linéaire) qui vous paraitra la plus pertinente en le justifiant.
  
  \item Après avoir justifié qu'une distribution normale représente des
  fluctuations aléatoire, discuter avec $\tau = 1$\,h et 24\,h les écarts de la
  pdf de la production électrique à cette loi pour les 3 sources.
\end{enumerate}

\section{Agrégation}

Pour une source de puissance particulière, l'agrégation consiste à étendre la production spatialement pour profiter de conditions climatiques le plus décorrélées possible (cas aléatoire) ou anticorrélées (cas déterministe)
\begin{enumerate}
	\item Expliquer pourquoi des conditions climatiques décorrélées ou anticorrélées sont avantageuses.
	
  \item À partir du programme \tmtextit{agregation.py} qui utilise des
  données issues de \tmtextit{RTE} décrivant la production électrique
  issues de deux régions, représenter puis additionner les sources
  électriques PV ou éolien.
  
  \item Discuter
  \begin{enumerate}
    \item Les séries temporelles obtenues. L'agrégation permet elle de
    diminuer les periodes sans puissance produite ?
    
    \item Comparer les distributions obtenues avec une loi normale
    
    \item Calculer la covariance $cov=\langle (X-\langle X \rangle)(Y - \langle Y \rangle)\rangle $ et le coefficient de corrélation $r=\frac{cov}{\sigma_X \sigma_Y}$ avec $X$ et $Y$ les productions de puissance instantanées des deux régions considérées. Conclure.
  \end{enumerate}
\end{enumerate}

\section{Foisonnement}

Le foisonnement consiste à multiplier les sources de production de puissance
à priori décorrélées pour tendre vers une distribution normale. Il s'agit de s'appuyer sur une conséquence du théorème central limite qui prédit un
comportement aléatoire d'une somme de variables elles même aléatoires. La
condition étant que les variables soient absolument décorrélées. En vous
appuyant sur le programme \tmtextit{foisonnement.py} additionner les sources
éolien et PV et représenter la série temporelle correspondante.

\begin{enumerate}
  \item À partir du programme \tmtextit{foisonnement.py}, aditionner les
  sources de production de puissance électrique PV et éolien d'un espace géographique que l'on précisera 
  
  \item Commenter
  \begin{enumerate}
    \item Les séries temporelles obtenues. L'agrégation permet elle de
    diminuer les periodes sans puissance produite ? La perte de puissance
    nocturne est elle atténuée ?
    
    \item Comparer les distributions obtenues avec une loi normale
    
    \item Calculer la covariance $cov=\langle (X-\langle X \rangle)(Y - \langle Y \rangle)\rangle $ et le coefficient de corrélation $r=\frac{cov}{\sigma_X \sigma_Y}$ avec $X$ et $Y$ les productions de puissance instantanées des sources considérées. Conclure.
    
  \end{enumerate}
\end{enumerate}

\section{Intermittence}

Dans une première approche de l'intermittence, nous allons évaluer la
capacité d'une source de production de puissance à la variation de la consommation. C'est une manière de combiner la prévisibilité de la variabilité et la fractionnabilité.

Pour cela en vous appuyant sur les programmes précédents, vous allez, vous allez calculer la corrélation entre les variations de la consommation, déterminée par le signe de $\left( \frac{C (t) - C (t + \tau)}{\langle C \rangle} \right)$ et celui des sources respectivement nucléaire, éolien et solaire.
On rappelle que le coefficient de corrélation entre les grandeurs aléatoires $x$ et $Y$, $r=\frac{cov}{\sigma_X \sigma_Y}$ dérive de la covariance $cov=\langle (X-\langle X \rangle)(Y - \langle Y \rangle)\rangle $.

Ajouter le calcul proposé au programme \tmtextit{correlation\_simple.py} puis
\begin{enumerate}
	\item commentez les valeurs obtenues, en faisant varier $\tau$.
	
	\item estimez les gains avec dans le cas du foisonnement et dans le cas de l'agrégation.
	
\end{enumerate}

%
%
%Pour cela en vous appuyant sur le programme \tmtextit{correlation\_simple.py},
%vous allez calculer la probabilité que le signe de la variation de la
%consommation, déterminée par le signe de $\left( \frac{C (t) - C (t +
%\tau)}{\langle C \rangle} \right)$ et celui des sources respectivement
%nucléaire, éolien et solaire soient les mêmes.
%\begin{enumerate}
%	\item Commentez les valeurs obtenues, en faisant varier $\tau$.
%	
%	\item Estimez les gains avec dans le cas du foisonnement et dans le cas de l'agrégation.
%	
%\end{enumerate}

\section{Conclure}

\end{document}
