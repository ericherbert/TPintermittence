\documentclass[12pt,french]{article}
\usepackage[utf8]{inputenc}
\usepackage[T1]{fontenc}
\usepackage{epsfig}
\usepackage{babel}
\usepackage{amsmath}
\usepackage{amsfonts}
\usepackage{fancyhdr}
\usepackage{fancybox}
\usepackage{vmargin}
\usepackage{wrapfig}
\usepackage{times}
\usepackage{enumerate}
\usepackage[table]{xcolor}
\pagestyle{fancyplain}
\usepackage{hyperref}
\usepackage{a4wide}
\usepackage{listings}
\lstset{
	basicstyle=\ttfamily,
	columns=fullflexible,
	frame=single,
	breaklines=true,
	postbreak=\mbox{\textcolor{red}{$\hookrightarrow$}\space},
}

\usepackage{easyReview}

\setmargins{2.5cm}{2cm}{16cm}{23.5cm}{0cm}{0cm}{1cm}{1cm}



\baselineskip 20mm
\headheight=0mm
\headsep=15mm
\setlength\textheight{230mm}

\newcommand{\re}{\color{red}}

\newcommand{\be}{\begin{equation}}
\newcommand{\ee}{\end{equation}}
\newcommand{\ba}{\begin{eqnarray}}
	\newcommand{\ea}{\end{eqnarray}}
\newcommand{\dal}{\raisebox{0.085cm}}
\newcommand{\dslash}{{\not\!\sectionial}}

\newcommand{\vv}{\overrightarrow}

%%%%%%%%%% Start TeXmacs macros
\newcommand{\tmtextbf}[1]{{\bfseries{#1}}}
\newcommand{\tmtextit}[1]{{\itshape{#1}}}
%%%%%%%%%% End TeXmacs macros


%\geometry{top=2cm, bottom=2cm, left=2cm , right=2cm}



\begin{document}

	
	
	
	\lhead[\fancyplain{}{\thepage}]{\fancyplain{}{Université Paris Cité 2024 -- 25}}
	\chead{}
	\rhead[\fancyplain{}{\rightmark}]{\fancyplain{}{M1 PFA -- module Énergie PF2BY110 }}
	
	%\cfoot{}
	
\begin{center}
    \begin{tabular}{c}
        {\Large  \textsc{Énoncé de TP}}\\
        {\Large{Intermittence de la production et de la consommation de puissance}}\\
        \\
        %\hline
    \end{tabular}
    
    \normalsize 
    
    \hrule
    \vspace{0.5cm}
    
    --- ATTENTION ---\\
    Pour accéder à la salle de TP vous devrez présenter le travail \\ 
    correspondant aux questions posées à la fin de la subsection~\ref{sec:avant}.
    
    \vspace{0.5cm}
    \hrule
\end{center}

\alert{pour 'lannée prochaine: \\
-- modifier fichiers horodatage en heure directement (pas de demie heure) \\
-- sur jupyterhub de l'ufr le calcul des correlations est trop long. voir sir la réduction au dessus suffit ou s'il faut réduire plus\\
-- vérifier que les correlations sont normalisées correctement. J'ai l'impression qu'il y a un problème dans les fichiers de données \\
-- refaire tout le TP pour vérifier que les changements sont ok}
\vspace{1cm}

Le compte rendu de ce TP devra être rendu sous forme numérique, en \textbf{pdf},
édité à l'aide d'un outil tel que \textit{Latex}, \textit{Texmacs} ou \textit{LibreOffice}. Les figures seront
commentées et choisies selon leur pertinence.
\textbf{Votre nom} et le \textbf{titre du TP}  devront apparaitre dans le titre de votre compte rendu.
Envoyer votre rapport à l'adresse  \href{mailto:eric.herbert@u-paris.fr}{eric.herbert@u-paris.fr}. \\
Vous allez générer dans ce TP un grand nombre de figures. N'oubliez pas de toujours les nommer de manière à pouvoir les retrouver, inscrire les axes et dimensions. Enfin vous pouvez (devez) opérer une sélection pour votre rapport. 


\section{Avant la séance}
\label{sec:avant}
L'intermittence est une notion très précise en physique, qui signifie que
l'on observe pour une variable aléatoire des fluctuations de grande amplitude dont la fréquence s'écarte de celle attendue pour une loi normale. Par exemple la distribution des incréments de vitesse dans un écoulement turbulent\footnote{On trouvera une abondante littérature sur le sujet à l'aide d'un moteur de recherche. Ce matériau pourra servir à l'introduction de votre rapport}  ne suit pas exactement la loi normale.
Cependant dans le débat lié à
la production de puissance, et notamment dans la comparaison entre les
énergie de flux (éolien, photovoltaïque, appelé PV dans la suite) et les énergies de stock (nucléaire,
fossile) dans leur capacité à répondre à une demande de puissance
électrique, cette définition n'est pas suffisante. En effet, par exemple,
une variation parfaitement prévisible est que le PV ne produira jamais de
puissance la nuit, et ne pourra donc jamais satisfaire une demande non nulle, quelle que
soit la taille du parc installé. Ainsi, il est nécessaire de tenir compte
des caractéristiques intrinsèques de la production et de la consommation et
de leur couplage, et d'identifier les notions suivantes:
\begin{itemize}
  \item \textbf{Variabilité pilotable} (modulation voulue de la puissance produite), prévisible (entretien), non prévisible (accident)
  
  \item \textbf{Prévisibilité de la production}. La nuit, pas de production PV, qui s'oppose aux fluctuations de la vitesse du vent
  
  \item \textbf{Fractionnabilité}, c'est à dire la possibilité de n'utiliser qu'une
  section de la puissance installée (il n'est pas possible d'arrêter une demie tranche de réacteur nucléaire, par contre il est possible de choisir un point de fonctionnement à $P_{max}/2$)  
\end{itemize}
Ces paramètres mènent à une grande confusion. Nous allons essayer dans ce
TP de percevoir les caractéristiques des différentes sources de production
de puissance électrique et d'en extraire une estimation simple de
l'intermittence.

Nous nous appuierons pour cela sur les données de production et d'échange de puissance électrique produites par RTE le Réseau de Transport Électrique français.


\subsection{Récupérer les données et les programmes}

\paragraph{Données brutes}
Remarques sur les données de RTE: \textbf{1/} EDF est tenu légalement d'acheter toutes la production électrique des sources PV et eolien, donc la puissance injectées de ces sources dans le réseau est toujours la puissance maximale disponible. \textbf{2/} le réseau est toujours équilibré, c'est à dire qu'à chaque instant la production totale est égale à la consommation totale. \textbf{3/} les source fossiles, nucléaire et hydraulique étant pilotables, elles s'adaptent à la variation de la consommation.

\begin{enumerate}
    \item Se rendre sur le site de RTE à l'adresse 
    \href{https://www.rte-france.com/}{https://www.rte-france.com/} puis naviguer vers l'onglet Production d'électricité par filière.\footnote{Sinon accessible ici \href{https://www.rte-france.com/eco2mix/la-production-delectricite-par-filiere}{https://www.rte-france.com/eco2mix/la-production-delectricite-par-filiere}. Un bouton \textit{téléchargement} est disponible en bas de la page.}
    
    \item Identifier les différentes sources de puissance, les flux exportés et importés.
    
    \item  Comparer qualitativement sur différentes périodes leur série temporelle.
    
    \item Quelles sont les valeurs typiques de production de puissance pour chaque source ?
    
    \item Faire un bilan comparatif des points précédents
\end{enumerate}

Dans la deuxième section, nous travaillerons avec un échantillon des données disponibles sur ce site qui ont été mises en page pour pouvoir être exploitées directement.  Elles sont dans le répertoire \tmtextit{DATA}, des du dépôt de ce TP.

\paragraph{Programmes}
Rendez vous sur la page \href{https://github.com/ericherbert/TPintermittence}{https://github.com/ericherbert/TPintermittence} et télécharger l'ensemble du projet (bouton \tmtextit{clone or download}) et extraire les
documents de l'archive. Vous y trouverez:
\begin{itemize}
  \item Le dossier \tmtextit{DATA} contenant les données téléchargées depuis le site de \tmtextit{RTE}. Ces données vous permettront de réaliser le TP. Vous pouvez également choisir de télécharger des données pour d'autres périodes, il faudra alors procéder à une mise en forme.
  
  \item L'énoncé du TP en pdf
  
  \item Des programmes \tmtextit{Python} permettant la réalisation de ce TP. Le programme \textit{outils.pdf} contient des fonctions pour calculer la covariance et la corrélation.
\end{itemize}
Ce TP est prévu pour être effectué sur python. Vous pouvez utiliser le \tmtextit{JupyterHub} de l'UFR vous permettra de travailler via votre explorateur web et sans rien installer sur votre machine. Vous la trouverez à l'adresse suivante
\href{https://jupy.physique.u-paris.fr/}{https://jupy.physique.u-paris.fr/}. Vous devez normalement pouvoir vous y connecter avec vos identifiants ENT.



¨
%Pour exploiter des données autres que celles proposées dans le dépôt github, vous pouvez suivre cette procédure: À présent nous allons explorer une période spécifique de production de puissance. Vous aurez également besoin de données situées dans des régions géographiques éloignées.
%  
%\begin{enumerate}  
%  \item Choisir une période et télécharger les données
%  correspondantes. Noter la durée, la fréquence d'échantillonnage, les
%  sources disponibles. 
%  
%  \item Déziper l'archive et enregistrer le
%  tableau au format csv, Prendre garde également aux caractères non   reconnus, séparateur décimal etc. Il peut être utile d'ouvir le fichier pour vérifier son contenu.
%  
%  \item Utiliser le programme \tmtextit{open\_data.py} que vous aurez modifié
%  pour votre usage, pour extraire les différentes données. Noter le nom et
%  la colonne de chaque mesurable (Consommation, nucléaire etc). 
%  
%\end{enumerate}




\section{Traitement}


\subsection{Intermittence}

Dans une première approche de l'intermittence, nous allons évaluer la capacité des sources de production de puissance à suivre la variation de la consommation.  Pour cela nous allons utiliser la notion de corrélation. Pour fixer les idées rapidement, nous allons construire un signal sinusoidal et calculer son autocorrélation: 
\begin{lstlisting}
	import numpy as np
	sin = np.sin(2*np.pi*np.arange(0,5,0.1))
	print(np.correlate(sin,sin))
	autocor = np.correlate( sin, sin, 'same' ) / np.std(sin) / np.std(sin) / len(sin)
\end{lstlisting}
Faites le plot de \texttt{autocor}. Commenter le plot obtenu, la valeur numérique affichée, que nous appelons le coefficient de corrélation $r$. Si besoin, appuyez vous sur l'aide de la fonction \texttt{np.autocor} et de la page \href{https://fr.wikipedia.org/wiki/Corr\%C3\%A9lation_(statistiques)}{Wikipedia} correspondante.

Dans la suite on s'interessera uniquement aux sources de production nucléaire, éolien et PV. L'expression \tmtextit{les 3 sources} s'y réfère.


\subsection{Représentations brutes}
Après avoir choisi un fichier disponible dans \textit{DATA}. Les fichiers sont régionalisés, sont disponibles l'Occitanie, les Hauts de France et toute la France.
\paragraph{Séries temporelles}
\begin{enumerate}
  \item À l'aide du programme \tmtextit{plot\_data.py}, représenter les séries temporelles de la \tmtextbf{consommation} journalière et saisonnière (par exemple 3 mois). Ajouter les unités et enregistrer ces figures de manière à pouvoir les retrouver simplement, et faites de même pour les \textbf{trois sources}. Vous pourrez commenter les points suivants:
  \begin{enumerate}
    \item Sur les diagrammes annuels, repérer les différentes saisons
    
    \item Sur les diagrammes journaliers, repérer les différentes sections de la journée
    
  \end{enumerate}
  \item Extraire les moyennes temporelles $\langle x \rangle$ et écart type
  $\sigma_x$ de chacune des sources sur une une semaine que vous aurez choisie.
  Quelles sont les limitations de l'emploi des quantités $\sigma_x $, $ \langle
  x \rangle$ et de la variation relative $\sigma_x / \langle
  x \rangle$, pour les caractériser ?
\end{enumerate}

\paragraph{Distributions} À l'aide du programme \tmtextit{plot\_distribution.py}, représenter la distribution de la \tmtextbf{consommation} et des \textit{3 sources}. Représenter les \tmtextit{densités de probabilité} \footnote{appelée pdf pour Probability Density Function, définie telle que $\int pdf = 1$} de ces fonctions en vous appuyant sur le paramètre \tmtextit{density} de la fonction histogramme. 

\paragraph{Autocorrélations}  À l'aide du programme \tmtextit{plot\_correlation.py}, représenter l'autocorrélation de la \tmtextbf{consommation} et des \textit{3 sources}. 
Ajouter les unités, noms des axes et enregistrer les figures. Pouvez vous identifier des motifs periodiques ?

\paragraph{Corrélations}  Calculer les coefficients de corrélation entre la consommation et \texttt{C} et chacune des trois sources \texttt{S} à l'aide de 
\begin{lstlisting}
	autocor = np.correlate(  C,  S ) / np.std(C) / np.std(S) / len(C)
\end{lstlisting}
conclure sur les capacités respectives de chacune des sources à satisfaire la demande.

Une manière d'améliorer cette capacité est de varier les sources, à la fois géographiquement (agrégation) et techniqument (foisonnement). C'est ce que nous allons voir dans la suite.


\subsection{Agrégation}

Pour une source de puissance particulière, l'agrégation consiste à étendre la production spatialement pour profiter de conditions climatiques le plus décorrélées possible (cas aléatoire) ou anticorrélées (cas déterministe)
\begin{enumerate}
	\item Expliquer pourquoi des conditions climatiques décorrélées ou anticorrélées pourraient être avantageuses.
	
  \item À partir du programme \tmtextit{agregation.py} qui utilise des  données issues de \tmtextit{RTE} décrivant la production électrique   issues de deux régions, représenter puis additionner les sources
  électriques PV \textbf{ou} éolien.
  
  \item Discuter :
  \begin{enumerate}
    \item Les séries temporelles obtenues. L'agrégation permet elle de diminuer les périodes sans puissance produite ?
        
    \item Calculer le coefficient de corrélation $r=\frac{cov}{\sigma_X \, \sigma_Y}$ avec $X$ et $Y$ les productions de puissance instantanées des deux régions considérées et la covariance $cov=\langle (X-\langle X \rangle)(Y-\langle Y \rangle) \rangle$.
    
    \item Conclure sur l'agrégation 
    
  \end{enumerate}
\end{enumerate}

\subsection{Foisonnement}

Le foisonnement consiste à multiplier les sources de production de puissance
à priori décorrélées pour tendre vers une distribution normale. Il s'agit de s'appuyer sur une conséquence du théorème central limite qui prédit un
comportement aléatoire d'une somme de variables elles même aléatoires. La
condition étant que les variables soient absolument décorrélées. En vous
appuyant sur le programme \tmtextit{foisonnement.py} additionner les sources
éolien et PV et représenter la série temporelle correspondante.

\begin{enumerate}
  \item À partir du programme \tmtextit{foisonnement.py}, aditionner les
  sources de production de puissance électrique PV et éolien d'un espace géographique que l'on précisera 
  
  \item Commenter
  \begin{enumerate}
    \item Les séries temporelles obtenues. L'agrégation permet elle de
    diminuer les periodes sans puissance produite ? La perte de puissance
    nocturne est elle atténuée ?
     
    \item Calculer le coefficient de corrélation $r=\frac{cov}{\sigma_X \, \sigma_Y}$ avec $X$ et $Y$ les productions de puissance instantanées des sources considérées. 
    
    \item Conclure le foisonnement
  \end{enumerate}
\end{enumerate}

Une autre manière de pouvoir satisfaire la demande est de disposer d'un stockage. C'est ce que nous allons voir dans la suite.

%
%
%Pour cela en vous appuyant sur le programme \tmtextit{correlation\_simple.py},
%vous aller calculer la probabilité que le signe de la variation de la
%consommation, déterminée par le signe de $\left( \frac{C (t) - C (t +
%\tau)}{\langle C \rangle} \right)$ et celui des sources respectivement
%nucléaire, éolien et solaire soient les mêmes.
%\begin{enumerate}
%	\item Commenter les valeurs obtenues, en faisant varier $\tau$.
%	
%	\item Estimer les gains avec dans le cas du foisonnement et dans le cas de l'agrégation.
%	
%\end{enumerate}



\section{Stockage et Conversion}
L'objet de cette section est d'évaluer et de dimensionner les solutions à mettre en oeuvre pour compenser l'intermittence intrinsèque de sources de puissance. On supposera que la puissance requise est toujours fournie et qu'elle correspond à la consommation.


%\begin{enumerate}
%	\item Tracer la série temporelle de la consommation de puissance sur une année $P_C(t)$
%	
%	\item Tracer les séries temporelles de la production de puissance photovoltaique (PV) $P_{PV}(t)$ et éolien $P_W(t)$ sur la même periode pour l'ensemble du territoire français.
%	
%	\item Tracer les séries temporelles des ratio $r_{PV} = P_C / P_{PV}$ et $r_{W} = P_C / P_{W}$.
%	
%	\item Discuter le sens de ces ratios, notamment la signification de $r=0$, $r=1$ et $r \rightarrow \infty$. 
%	
%	\item Discuter les possibilités de compenser les variations non désirées de production de puissance, dues aux accidents et à la variabilité naturelle pour les source de type flux (PV, éolien etc), et pour les accidents pour une source de type potentiel (petrole, gaz, nucléaire etc) notamment stockage et surdimensionnement.
%	
%	\item Conclure sur les besoins en stockage de chacune de ces sources. 
%\end{enumerate}

\subsection{Stockage}
Nous allons dans cette section normaliser la quantité d'énergie produite sur une année par la quantité d'énergie consommée dans cette même année pour ne plus dépendre de la taille du parc de production de puissance. \\
Les questions suivantes sont à discuter pour les sources PV et éolien.
\begin{enumerate}
	\item Quantifier $E_0^C$ la totalité de l'énergie consommée dans l'année choisie. 
	
	\item Porcédez de même pour quantifier les énergies $E_0^{PV}$ et $E_0^{W}$ produites sur l'année en France.
	
	\item Tracer la série temporelle de la différence des puissances normalisées pour le PV et l'éolien $\delta^{PV} (t) = \frac{P(t)}{E_0^{W}} - \frac{C(t)}{E_0^C}$ et l'éolien $\delta^{W}(t) = \frac{P(t)}{E_0^{W}} - \frac{C(t)}{E_0^C}$. 
	
	\item Vérifier que dans les deux cas $\sum_0^{1\,y} \delta(t) ~ \text{d} t $ sur un an est bien nul. \\
	Vous vous êtes ainsi assuré de produire exactement sur l'année la quantité d'énergie nécessaire à la consommation, sans sur-production ni sous-production. Cependant la synchronisation de la consommation et de la production ne sont pas assurées. Pour cela il faut que consommation et production soient égaux à chaque pas de temps.\\ 
	En vous appuyant sur les $\delta$, concluez si c'est bien le cas.
	
	\item Nous allons à présent caculer la taille du stockage nécessaire. \\
	Tracer la série temporelle de la différence des sommes cumulées $\Delta(t) = \sum \delta (t)$. $\Delta $ représente une estimation du stockage nécessaire à chaque pas de temps, en unité de $E_0^C$.  totalité de la production  Son maximum est le stockage nécessaire. Si $\Delta_{MAX})=0.5$ alors il faut pouvoir stocker la moitité de la consommation annuelle.
	
	\item En déduire une évaluation du volume de stockage $S$ nécessaire en fraction de la consommation annuelle $E_0^C$.
	
	\item Question subsidiaire: comment $S$ dépend de la période choisie ?
	
\end{enumerate}



%Puis  représenter la somme cumulée et normalisée de la consommation de puissance sur cette année:
%\begin{equation}
%	C^*(T)  = \frac{1}{C_0} ~ \int_{0}^{T}P_C(t) \; \text{d} t	
%\end{equation}
%Cette expression représente l'énergie produite entre $t=0$ et $t=T$. Par construction vous devez vérifier que $C^*(0)=0$ et $C^*(T=1)=1$ avec $T=1$ correspondant à un an. 
%
%Procéder de la même manière que pour la consommation et quantifier la totalité de l'énergie produite dans l'année choisie $E_0$. Représenter la somme cumulée et normalisée de la consommation sur cette année: 
%\begin{equation}
%	E^*(t)  = \frac{1}{E_0} ~ \int_{0}^{T} P(t) \; \text{d} t	
%\end{equation}
%Verifier que $E^*(0)=0$ et $E^*(T=1)=1$ avec $T=1$ correspondant à un an. 


\subsection{Conversion en énergie}

Vous allez à présent estimer la quantité d'énergie et le volume correspondant à la quantité $S$ calculée précédemment. 

\begin{enumerate}
	\item Convertir le volume de stockage normalisé trouvé dans la section précédente en une unité d'énergie.
	\item La solution du dihydrogène est souvent mis en avant pour les grands volumes de stockable, expliquer pourquoi.
	\item Convertir $S$ en volume et en masse d'hydrogène, en prenant des ordres de grandeur de la littérature.
\end{enumerate}


\subsection{Conclure}
\begin{enumerate}
	\item Conclure sur les contraintes imposées par une production de puissance de type flux (quelle qu'elle soit). 
	
	\item Comparer les deux  sources PV et éolien
\end{enumerate}




\end{document}
