\documentclass{article}
\usepackage[french]{babel}
\usepackage[utf8]{inputenc}
\usepackage{hyperref}

%%%%%%%%%% Start TeXmacs macros
\newcommand{\tmtextbf}[1]{{\bfseries{#1}}}
\newcommand{\tmtextit}[1]{{\itshape{#1}}}
%%%%%%%%%% End TeXmacs macros

\begin{document}

\title{TP intermittence}

\maketitle

Le compte rendu de ce TP devra être rendu sous forme numérique, en pdf, à
l'aide d'un outil tel qu'un latex ou LibreOffice. Les figures seront
commentées et choisies selon leur pertinence.

\section{Avant de commencer}

\begin{itemize}
  \item On s'intéresse à la production de puissance électrique en France.
  Légalement EDF doit acheter toutes la production PV et eolien, donc la
  puissance de ces sources injectées dans le réseau est toujours la maximale
  possible. Inversement, les source fossiles étant pilotables, elles
  s'adaptent à la variation de la consommation.
  
  \item L'intermittence n'est pas réductible à la
  \begin{itemize}
    \item Variabilité, pilotable ou non, prévisible ou non de la production
    de puissance
    
    \item Prévisibilité,
    
    \item Fractionnabilité, c'est à dire la possibilité de n'utiliser une
    partie de sa 
  \end{itemize}
\end{itemize}
L'intermittence ne se défini pas simplement.

\section{Récupérer les données et les programmes}

\subsection{Programmes}

rendez vous sur la page
\href{}{https://www.rte-france.com/fr/eco2mix/eco2mix-mix-energetique},
téléchargez l'ensemble du projet (bouton \tmtextit{clone or download}) et
extraire les documents de l'archive. Vous y trouverez
\begin{itemize}
  \item Le dossier \tmtextit{DATA} contenant les données dont vous avez
  besoin si la phase de téléchargement des données sur le site de
  \tmtextit{RTE} n'a pas fonctionné
  
  \item L'énoncé en pdf
  
  \item Des programmes \tmtextit{Python} permettant la réalisation de ce TP.
\end{itemize}
Ce TP doit normalement être effectué sur l'application \tmtextit{JupyterHub}
de l'UFR. Vous le trouverez à l'adresse suivante
\href{https://jupy.physique.univ-paris-diderot.fr/}{https://jupy.physique.univ-paris-diderot.fr/}.

\subsection{Données}

\begin{enumerate}
  \item Se rendre sur le site de RTE à l'adresse
  \href{}{https://www.rte-france.com/fr/eco2mix/eco2mix-mix-energetique}
  
  \item Identifier les différentes sources de puissance, comparer
  qualitativement sur différentes periodes leur série temporelle.
  
  \item Choisir une periode spécifique et télécharger les données
  correspondantes. Noter la durée, la fréquence d'échantillonnage, les
  sources disponibles.
  
  \item Après avec dézippé l'archive et enregistré le tableau au format
  csv, utiliser le programme \tmtextit{open\_data.py} que vous aurez modifié
  pour votre usage, pour extraire les différentes données. Noter le nom et
  la colonne de chaque mesurable (Consommation, nucléaire etc). Si besoin
  dans le repertoire \tmtextit{DATA}, des données provenant de RTE sont
  disponibles.
\end{enumerate}

\section{Traitement}

Dans la suite on s'interesera uniquement aux sources de production nucléaire,
éolien et solaire. L'expression \tmtextit{les 3 sources} s'y réfère

\subsection{Séries temporelles}

\begin{enumerate}
  \item À l'aide du programme \tmtextit{plot\_data.py}, représenter la
  dynamique temporelle de la \tmtextbf{consommation} journalière, mensuelle
  et annuelle. Ajoutez les unités et enregistrez ces figures de manière à
  pouvoir les retrouver simplement, et faites de même pour les trois sources.
  Ajoutez les unités et enregistrez ces figures de manière à pouvoir les
  retrouver simplement.
  \begin{enumerate}
    \item Sur les diagrammes annuels, repérez les différentes saisons
    
    \item Sur les diagrammes journaliers, repérez les différentes parties de
    la journée
  \end{enumerate}
  \item Extraire les moyennes temporelles $\langle x \rangle$ et écart type
  $\sigma_x$ de chacune des sources sur chacune des periodes considérées.
  Pouvez vous utiliser les quantités $\sigma_x / \langle x \rangle$, pour
  caractériser chacune des sources ?
\end{enumerate}

\subsection{Distribution des séries temporelles}

\begin{enumerate}
  \item À l'aide du programme \tmtextit{plot\_distribution.py}, représenter
  la variation de la distribution de la \tmtextbf{consommation} normalisée
  par sa moyenne temporelle $\frac{C (t) - C (t + \tau)}{\langle C \rangle}$,
  avec un pas de temps $\tau$ d'une heure puis de 24 heures. Faire de même
  pour les 3 sources. Représenter les \tmtextit{densités de probabilité}
  (pdf) de ces fonctions en vous appuyant sur le paramètre \tmtextit{density}
  de la fonction histogramme, puis comparer ces distributions avec une loi
  normale (Gaussienne) de même moyenne et écart type. Ajouter les unités,
  noms des axes et enregistrer les figures. Vous choisirez la représentation
  (logarithmique ou linéaire) qui vous paraitra la plus pertinente.
  
  \item Discuter à 1h et 24h les écarts à la loi normale.
\end{enumerate}

\section{Agrégation}

L'agrégation consiste à séparer les sources de puissances de même nature
pour profiter de conditions climatiques le plus décorrélées possible.
\begin{enumerate}
  \item À partir du programme \tmtextit{agregation.py} qui utilise des
  données issues de \tmtextit{RTE} décrivant la production électrique
  issues de deux régions, représenter puis aditionner les sources
  électriques PV ou éolien.
  
  \item Commenter la correlation des productions éléctriques dans ce cas
\end{enumerate}

\section{Foisonnement}

Le foisonnement consiste à multiplier les sources de production de puissance
à priori décoreler pour tendre vers une distribution. en vous appuyant sur
le programme \tmtextit{foisonnement.py} aditionner les sources éolien et PV
et retpésenter la série temporelle correspondante.
\begin{enumerate}
  \item À partir du programme \tmtextit{foisonnement.py}, aditionner les
  sources de production de puissance électrique PV et éolien.
  
  \item Refaire les \tmtextit{pdf} des séries temporelles.
  
  \item Que peut on observer ? La perte de puissance nocturne est elle
  atténuée ?
  
  \item Commenter la correlation des productions éléctriques dans ce cas
\end{enumerate}

\section{Intermittence}

Dans une première approche de l'intermittence, nous pouvons l'évaluer comme
la capacité à s'adapter à la variation de la consommation. Pour cela en
vous appuyant sur le programme \tmtextit{correlation\_simple.py}, vous allez
calculer la probabilité que le signe de la variation de la consommation,
déterminée par le signe de $\left( \frac{C (t) - C (t + \tau)}{\langle C
\rangle} \right)$ et celui des sources respectivement nucléaire, éolien et
solaire soient les mêmes.

Commentez les valeurs obtenues, en faisant également varier $\tau$

\

\end{document}
