\documentclass{article}
\usepackage[french]{babel}
\usepackage[utf8]{inputenc}
\usepackage{hyperref}

%%%%%%%%%% Start TeXmacs macros
\newcommand{\tmtextbf}[1]{{\bfseries{#1}}}
\newcommand{\tmtextit}[1]{{\itshape{#1}}}
%%%%%%%%%% End TeXmacs macros

\begin{document}

\title{TP intermittence}

\maketitle

Le compte rendu de ce TP devra être rendu sous forme numérique, en pdf,
édité à l'aide d'un outil tel que latex ou LibreOffice. Les figures seront
commentées et choisies selon leur pertinence.

\section{Avant de commencer}

L'intermittence est une notion très précise en physique, qui signifie que
l'on observe pour une variable aléatoire des fluctuations de grande amplitude
qui s'écartent de la loi normale. Cependant dans le débat publique lié à
la production de puissance, et notamment dans la comparaison entre les
énergie de flux (éolien solaire) et les énergies de stock (nucléaire,
fossile) dans leur capacité à répondre à une demande de puissance
électrique, cette définition n'est pas suffisante. En effet, par exemple,
une variation parfaitement prévisible est que le PV ne produira jamais de
puissance la nuit, et ne pourra donc jamais satisfaire une demande quel que
soit la taille du parc installé. Ainsi, il est nécessaire de tenir compte
des caractéristiques intrinsèques de la production et de la consommation et
de leur couplage, et d'identifier les notions suivantes:
\begin{itemize}
  \item Variabilité, pilotable et/ou prévisible (arret pour entretien) ou
  non (arret pour accident)
  
  \item Prévisibilité de la production (la nuit, pas de production PV)
  
  \item Fractionnabilité, c'est à dire la possibilité de n'utiliser qu'une
  partie de la production 
\end{itemize}
Ces paramètres mènent à une grande confusion. Nous allons essayer dans ce
TP de percevoir les caractéristiques des différentes sources de production
de puissance électrique et d'en extraire une estimation simple de
l'intermittence.

\section{Récupérer les données et les programmes}

\subsection{Programmes}

Rendez vous sur la page
\href{}{https://www.rte-france.com/fr/eco2mix/eco2mix-mix-energetique},
téléchargez l'ensemble du projet (bouton \tmtextit{clone or download}) et
extraire les documents de l'archive. Vous y trouverez
\begin{itemize}
  \item Le dossier \tmtextit{DATA} contenant les données dont vous avez
  besoin si la phase de téléchargement des données sur le site de
  \tmtextit{RTE} n'a pas fonctionné
  
  \item L'énoncé en pdf
  
  \item Des programmes \tmtextit{Python} permettant la réalisation de ce TP.
\end{itemize}
Ce TP doit normalement être effectué sur l'application \tmtextit{JupyterHub}
de l'UFR. Vous le trouverez à l'adresse suivante
\href{https://jupy.physique.univ-paris-diderot.fr/}{https://jupy.physique.univ-paris-diderot.fr/}.
Vous devez normalement pouvoir vous y connecter avec vos identifiants ENT. En
cas de problème, m'écrire.

Remarque sur les données de RTE: Légalement EDF doit acheter toutes la
production PV et eolien, donc la puissance de ces sources injectées dans le
réseau est toujours la maximale possible. Inversement, les source fossiles,
nucléaire et hydraulique étant pilotables, elles s'adaptent à la variation
de la consommation.

\subsection{Données}

\begin{enumerate}
  \item Se rendre sur le site de RTE à l'adresse
  \href{}{https://www.rte-france.com/fr/eco2mix/eco2mix-mix-energetique}
  
  \item Identifier les différentes sources de puissance, comparer
  qualitativement sur différentes periodes leur série temporelle.
  
  \item Choisir une periode spécifique et télécharger les données
  correspondantes. Noter la durée, la fréquence d'échantillonnage, les
  sources disponibles. Après avec dézippé l'archive et enregistré le
  tableau au format csv, Prendre garde également aux caractères non
  reconnus.
  
  \item Utiliser le programme \tmtextit{open\_data.py} que vous aurez modifié
  pour votre usage, pour extraire les différentes données. Noter le nom et
  la colonne de chaque mesurable (Consommation, nucléaire etc). Si besoin
  dans le repertoire \tmtextit{DATA}, des données provenant de RTE sont
  disponibles.
\end{enumerate}

\section{Traitement}

Dans la suite on s'interessera uniquement aux sources de production
nucléaire, éolien et solaire. L'expression \tmtextit{les 3 sources} s'y
réfère

\subsection{Séries temporelles}

\begin{enumerate}
  \item À l'aide du programme \tmtextit{plot\_data.py}, représenter la
  dynamique temporelle de la \tmtextbf{consommation} journalière, mensuelle
  et annuelle. Ajoutez les unités et enregistrez ces figures de manière à
  pouvoir les retrouver simplement, et faites de même pour les trois sources.
  Ajoutez les unités et enregistrez ces figures de manière à pouvoir les
  retrouver simplement.
  \begin{enumerate}
    \item Sur les diagrammes annuels, repérez les différentes saisons
    
    \item Sur les diagrammes journaliers, repérez les différentes parties de
    la journée
  \end{enumerate}
  \item Extraire les moyennes temporelles $\langle x \rangle$ et écart type
  $\sigma_x$ de chacune des sources sur chacune des periodes considérées.
  Quelles sont les limitations de l'emploi des quantités $\sigma_x / \langle
  x \rangle$, pour caractériser chacune des sources ?
\end{enumerate}

\subsection{Distribution des séries temporelles}

\begin{enumerate}
  \item À l'aide du programme \tmtextit{plot\_distribution.py}, représenter
  la variation de la distribution de la \tmtextbf{consommation} normalisée
  par sa moyenne temporelle $\frac{C (t) - C (t + \tau)}{\langle C \rangle}$,
  avec un pas de temps $\tau$ d'une heure puis de 24 heures. Faire de même
  pour les 3 sources. Représenter les \tmtextit{densités de probabilité}
  (pdf) de ces fonctions en vous appuyant sur le paramètre \tmtextit{density}
  de la fonction histogramme, puis comparer ces distributions avec une loi
  normale (Gaussienne) de même moyenne et écart type. Ajouter les unités,
  noms des axes et enregistrer les figures. Vous choisirez la représentation
  (logarithmique ou linéaire) qui vous paraitra la plus pertinente.
  
  \item Après avoir justifié qu'une distribution normale représente des
  fluctuations aléatoire, discuter avec $\tau = 1 h$ et 24h les écarts de la
  pdf de la production électrique à cette loi pour les 3 sources.
\end{enumerate}

\section{Agrégation}

L'agrégation consiste à séparer les sources de puissances de même nature
pour profiter de conditions climatiques le plus décorrélées possible.
\begin{enumerate}
  \item À partir du programme \tmtextit{agregation.py} qui utilise des
  données issues de \tmtextit{RTE} décrivant la production électrique
  issues de deux régions, représenter puis aditionner les sources
  électriques PV ou éolien.
  
  \item Commenter
  \begin{enumerate}
    \item Les séries temporelles obtenues. L'agrégation permet elle de
    diminuer les periodes sans puissance produite ?
    
    \item Comparer les distributions obtenues avec une loi normale
  \end{enumerate}
\end{enumerate}

\section{Foisonnement}

Le foisonnement consiste à multiplier les sources de production de puissance
à priori décorrélées \ pour tendre vers une distribution. Il s'agit de s
'appuyer sur une conséquence du théorème central limite qui prédit un
comportemnet aléatoire d'une somme de variables elle même aléatoires. La
condition étant que les varaibles soient absolument décorrélées. En vous
appuyant sur le programme \tmtextit{foisonnement.py} aditionner les sources
éolien et PV et retpésenter la série temporelle correspondante.
\begin{enumerate}
  \item À partir du programme \tmtextit{foisonnement.py}, aditionner les
  sources de production de puissance électrique PV et éolien.
  
  \item Commenter
  \begin{enumerate}
    \item Les séries temporelles obtenues. L'agrégation permet elle de
    diminuer les periodes sans puissance produite ? La perte de puissance
    nocturne est elle atténuée ?
    
    \item Comparer les distributions obtenues avec une loi normale
  \end{enumerate}
\end{enumerate}

\section{Intermittence}

Dans une première approche de l'intermittence, nous allons évaluer la
capacité à s'adapter à la variation de la consommation. C'est une manière
de combiner la prévisibilité de la variabilité et la fractionnabilité.
Pour cela en vous appuyant sur le programme \tmtextit{correlation\_simple.py},
vous allez calculer la probabilité que le signe de la variation de la
consommation, déterminée par le signe de $\left( \frac{C (t) - C (t +
\tau)}{\langle C \rangle} \right)$ et celui des sources respectivement
nucléaire, éolien et solaire soient les mêmes.

Commentez les valeurs obtenues, en faisant varier $\tau$.

Estimez les gains avec dans le cas du foisonnement et dans le cas de
l'agrégation.

\

\end{document}
